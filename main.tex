\documentclass[stu,12pt]{apa7}
  \usepackage{times}               % Times New Roman Font Face
  \usepackage[american]{babel}     % Localization
  \usepackage[utf8]{inputenc}      % Input Encoding
  \usepackage{hyperref}            % Hyperlinks
  \usepackage{enumitem}            % Additional Enumeration Environment Settings
  \usepackage{geometry}            % Page Layout
  \usepackage{soul}                % Text Highlighting
  \usepackage{graphicx}            % Images
  \usepackage{csquotes}            % Quoting Environment
  \usepackage{bookmark}            % Required by `csquotes'
  \usepackage{mdframed}            % Colorful Tex-Box Environment
  \usepackage[toc]{appendix}       % Appendix
  \usepackage{fancyhdr}            % Headings and Footers
  \usepackage[%
    style=apa,%
    sortcites=true,%
    sorting=nyt%
  ]{biblatex}
  \usepackage{xcolor}

  % Bibliography Setup
  %% Language Mappings
  \DeclareLanguageMapping{english}{english-apa}
  \DeclareLanguageMapping{american}{american-apa}
  %% Bibliography File Path
  \addbibresource{main.bib}
  %% Categories for Specified Bibliography Items
  %%% Category for sources not referenced in-text
  \DeclareBibliographyCategory{consulted}
  \addtocategory{consulted}{noauthor_business_nodate}
  \addtocategory{consulted}{noauthor_college_nodate}
  \addtocategory{consulted}{cline_science_2013}
  \addtocategory{consulted}{scire_breaking_2001}

  % Hyperlink Setup
  \hypersetup{
    colorlinks = true,
    urlcolor = blue,
    linkcolor = blue,
    citecolor = blue
  }

  % Page and Text Layout
  \geometry{%
    a4paper,%
    top=1in,%
    bottom=1in,%
    left=1in,%
    right=1in%
  }
  \setlength{\headheight}{15pt}

  % Header
  \lhead{COM120CG1-M3E}

  % Title Page
  \title{%
    Do You Hear What I Hear?
  }
  \shorttitle{Module 3 Essay Assignment}
  \author{Ashton Hellwig}
  \authorsaffiliations{Department of Mathematics, Front Range Community College}
  \course{COM125: Interpersonal Communication}
  \professor{Richard Thomas}
  \duedate{November 22, 2020 23:59:59 MDT}
  \date{\today}
  \abstract{%
    \textbf{Overview}\\%
    Some people believe that you either know how to listen, or you don’t.\\%

    However, like most skills, listening can be learned. In order to do so, you
      need to better understand what listening is and how it works. Sometimes we
      aren’t listening because we are busy talking and disclosing things about
      ourselves that should perhaps remain private.\\%

    Because listening is a multifaceted skill, there are many different kinds of
      listening and ways to listen. In this assignment, you will study the
      different types of listening and analyze what you observed in the Erin
      Brokovich movie clip titled, “The Bonus Check” which appears immediately
      below the topic titled, ``Skills of Effective Listening''. View the other
      videos as well.\\%

    You should spend approximately 6.5 hours on this assignment.%
  }


\begin{document}
  % Title Page
  \maketitle


  \section{What Is ``Listening'' and Why is it Important?}
    \subsection{The Types of Listening}
      \begin{description}
        \item[Discriminitive Listening]
          Discrimitive Listening ocurrs at the beginning receiving stages of
            hearing new auditory stimuli and is a more focus-intensive form
            of listening to our surroundings
            \parencite[pp. 333]{noauthor_communication_2013}.
          \begin{itemize}
            \item \textit{example}: After hearing a noise down the street come
              out of some alley we will focus our listening on that alley to
              ensure that the sound was not a sign of a struggle occurring, or
              an unknown danger about to occur.
          \end{itemize}
        \item[Informational Listening]
          Informational listening is when the listener is taking in the
            information with the intentions of comprehension and retention
            \parencite[pp. 334]{noauthor_communication_2013}.
          \begin{itemize}
            \item \textit{example}: Class lectures, tutorial videos
          \end{itemize}
        \item[Critical Listening]
          Critical listening involves the listener hearing the information being
            given to them with the goal of afterwards analysing and evaluating
            said information
            \parencite[pp. 334--335]{noauthor_communication_2013}. This
            typically involves the listener providing some sort of evaluation or
            judgement after the conversation (agreeing, disagreeing, etc).
          \begin{itemize}
            \item \textit{example}: Debate mediation, crisis counseling.
          \end{itemize}
        \item[Empathetic Listening]
          Empathetic listening is, in my opinion, the most important form of
            listening available. It involves us attempting to put ourselves into
            or at least feel the same emotions or pain that others are feeling
            \parencite[pp. 335--336]{noauthor_communication_2013}.
          \begin{itemize}
            \item \textit{example}: Therapy, friends talking about problems,
              crisis counseling, greif counseling.
          \end{itemize}
      \end{description}

    \subsection{Self Disclosure}
      Self Disclosure is when someone intentionally reveals their own private
        personal information. This can be positive or negative in connotation.
        For example, if someone is wearing a lapel pin which shows their
        political affiliation, this discloses that information about their
        political views to the public.

    \subsection{Listening's Role in Understanding}
      Listening, as we have already said previously, plays an incredibly
        important role in understanding (i.e.\ the \textit{most} important role).
        Listening is key to communication in any sort of relationship as it
        helps to form the bond between the speaker and the listener. People,
        and really \textit{any living being on this planet}, simply want to feel
        as though they are heared and understood when they speak to others. This
        also goes hand-in-hand with the skill of empathy. Empathy is, in my
        opinion, \textbf{the} most important trait that anyone can hold when
        communicating with others.

      Outside of our personal and business-related relationships, listening
        still plays an all-important role in our daily lives. As children
        all the way through our education we have to know to listen effectively
        to our teachers, guidance counselers, and professors. Anyone who has
        sat through a guest lecture or had a teacher with a
        less-than-enthusiastic way of speaking knows how difficult it is to
        listen to any topic that we do not personally care about and actually
        retain that information, which is why listening skills are imporant to
        develop as they can help someone overcome an obstical they may be facing
        by ``bouncing ideas off'' of their peers, which leads to a much better
        overall understanding of a topic by everyone involved
        \parencite[pp. 5]{caspersz_can_nodate}.

      It is a known issue in the academic community that listening is an
        \textit{incredibly} important skill for anyone to have a good grasp of.
        Unfortunately, this is not how schools tend to be teaching cirriculum
        and much more focus is spent on the acts of speaking, reading, and
        writing \parencite[pp. 3]{bodie_listening_2012}. To make the situation
        worse, when listening \textit{is} taught in the our education it is
        focused almost entirely on a classroom environment setting, forcing
        anyone in the class to believe that listening is simply just for
        quick and efficient information processing (i.e.\ during lectures or
        class presentations) \parencite{bostrom_rethinking_2011}.

    \subsection{The Most Important Type of Listening}
      The most important type of listening would have to be
        \textit{empathetic listening}. I believe this is important because
        empathy is, personally, what I like to look for depending on the type
        of problem I am coming to someone with. Empathy allows us to feel like
        we are not alone in our pain or happiness, and sharing the same emotion
        during a conversation as the speaker shows a deeper form of
        understanding than simply regurgitating the information.

    \subsection{Analysis of Erin's Listening in the Exploration Scene}
      I personally do not think that Erin's listening was the only problem
        that caused the misunderstanding we had witnessed, and I also do not
        believe that the misunderstanding was entirely due to a difference in
        the gender of the two individuals having the conversation. Erin listened
        \textit{just fine} in my opinion, as it is her boss who had mislead her
        by using somber tones and body language so as to suggest exactly what
        she had expected. Thus, Erin should really \emph{not} be embarassed
        by the encounter because clearly it was her boss who was just trying to
        ``poke-fun'' at her. Her boss also clearly knew that is exacly how she
        was going to react, judging by his expression when Erin reacted the way
        she did \parencite{soderbergh_erin_2000}.

    \subsubsection{Self-Disclosure's Role in Erin's Listening Methods}
      In the film, self-disclosure played a huge role in Erin's conflict with
        her boss. Because she had thought that she was not being appreciated
        for all of her hard work she ``lashed-out'' (which I believe to be
        an opinion but for the purposes of this paper we will continue). Self-
        disclosure is in all likelyhood what got her the larger salary in the
        first place --- many of those in media, business, and journalism tend
        to admire the individuals who ``stand out'' and hold the confidence
        required to make it in these competitive industries.
        \parencite{soderbergh_erin_2000}.


  % Bibliography
  %% Works Cited
  \newpage
  \printbibliography[%
    title={References},%
    heading={bibintoc},%
    notcategory={consulted}%
  ]


  %% Works Consulted
  \newpage
  \nocite{*}
  \printbibliography[%
    title={Additional References},%
    heading={bibintoc},%
    category={consulted}%
  ]
\end{document}
