\documentclass[stu,12pt]{apa7}
  \usepackage{times}               % Times New Roman Font Face
  \usepackage[american]{babel}     % Localization
  \usepackage[utf8]{inputenc}      % Input Encoding
  \usepackage{hyperref}            % Hyperlinks
  \usepackage{enumitem}            % Additional Enumeration Environment Settings
  \usepackage{geometry}            % Page Layout
  \usepackage{soul}                % Text Highlighting
  \usepackage{graphicx}            % Images
  \usepackage{csquotes}            % Quoting Environment
  \usepackage{bookmark}            % Required by `csquotes'
  \usepackage{mdframed}            % Colorful Tex-Box Environment
  \usepackage[toc]{appendix}       % Appendix
  \usepackage{fancyhdr}            % Headings and Footers
  \usepackage[%
    style=apa,%
    sortcites=true,%
    sorting=nyt%
  ]{biblatex}
  \usepackage{xcolor}

  % Bibliography Setup
  %% Language Mappings
  \DeclareLanguageMapping{english}{english-apa}
  \DeclareLanguageMapping{american}{american-apa}
  %% Bibliography File Path
  \addbibresource{main.bib}
  %% Categories for Specified Bibliography Items
  %%% Category for sources not referenced in-text
  \DeclareBibliographyCategory{consulted}
  \addtocategory{consulted}{noauthor_business_nodate}
  \addtocategory{consulted}{noauthor_college_nodate}
  \addtocategory{consulted}{soderbergh_erin_2000}
  \addtocategory{consulted}{cline_science_2013}
  \addtocategory{consulted}{bodie_listening_2012}

  % Hyperlink Setup
  \hypersetup{
    colorlinks = true,
    urlcolor = blue,
    linkcolor = blue,
    citecolor = blue
  }

  % Page and Text Layout
  \geometry{%
    a4paper,%
    top=1in,%
    bottom=1in,%
    left=1in,%
    right=1in%
  }
  \setlength{\headheight}{15pt}

  % Header
  \lhead{COM120CG1-M3E}

  % Title Page
  \title{%
    Do You Hear What I Hear?
  }
  \shorttitle{Module 3 Essay Assignment}
  \author{Ashton Hellwig}
  \authorsaffiliations{Department of Mathematics, Front Range Community College}
  \course{COM125: Interpersonal Communication}
  \professor{Richard Thomas}
  \duedate{November 22, 2020 23:59:59 MDT}
  \date{\today}
  \abstract{%
    \textbf{Overview}\\%
    Some people believe that you either know how to listen, or you don’t.\\%

    However, like most skills, listening can be learned. In order to do so, you
      need to better understand what listening is and how it works. Sometimes we
      aren’t listening because we are busy talking and disclosing things about
      ourselves that should perhaps remain private.\\%

    Because listening is a multifaceted skill, there are many different kinds of
      listening and ways to listen. In this assignment, you will study the
      different types of listening and analyze what you observed in the Erin
      Brokovich movie clip titled, “The Bonus Check” which appears immediately
      below the topic titled, ``Skills of Effective Listening''. View the other
      videos as well.\\%

    You should spend approximately 6.5 hours on this assignment.%
  }


\begin{document}
  % Title Page
  \maketitle


  \section{What Is ``Listening'' and Why is it Important?}
    There are \textit{many} types of listening we could delve into in-depth.
      For the purposes of this paper, we will talk about five (5) of them.

    \subsection{The Types of Listening}
      \begin{description}
        \item[Discriminitive Listening]
          Discrimitive Listening ocurrs at the beginning receiving stages of
            hearing new auditory stimuli and is a more focus-intensive form
            of listening to our surroundings
            \parencite[pp. 333]{noauthor_communication_2013}.
          \begin{itemize}
            \item \textit{example}: After hearing a noise down the street come
              out of some alley we will focus our listening on that alley to
              ensure that the sound was not a sign of a struggle occurring, or
              an unknown danger about to occur.
          \end{itemize}
        \item[Informational Listening]
          Placeholder.
            \parencite[pp. 334]{noauthor_communication_2013}
          \begin{itemize}
            \item \textit{example}: Placeholder.
          \end{itemize}
        \item[Critical Listening]
          Placeholder.
            \parencite[pp. 334--335]{noauthor_communication_2013}
          \begin{itemize}
            \item \textit{example}: Placeholder.
          \end{itemize}
        \item[Empathetic Listening]
          Placeholder.
            \parencite[pp. 335--336]{noauthor_communication_2013}
          \begin{itemize}
            \item \textit{example}: Placeholder.
          \end{itemize}
      \end{description}

    \subsection{Self Disclosure}
      Placeholder. Define Self-Disclosure in your own words. Placeholder.

    \subsection{Listening's Role in Understanding}
      Placeholder.

    \subsection{The Most Important Type of Listening}
      Placeholder.

    \subsection{Analysis of Erin's Listening in the Exploration Scene}
      Placeholder.

      \subsubsection{Self-Disclosure's ROle in Erin's Listening Methods}
      Placeholder.


  \section{Conclusion}
    Placeholder.


  % Bibliography
  %% Works Cited
  \newpage
  \printbibliography[%
    title={References},%
    heading={bibintoc},%
    notcategory={consulted}%
  ]


  %% Works Consulted
  \newpage
  \nocite{*}
  \printbibliography[%
    title={Additional References},%
    heading={bibintoc},%
    category={consulted}%
  ]
\end{document}
